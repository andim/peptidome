\documentclass[superscriptaddress,onecolumn,pre]{revtex4}
\bibliographystyle{apsrev}

\usepackage{ifthen}
\newboolean{pnas}
\setboolean{pnas}{false}

\usepackage{amsmath}
\usepackage{amsfonts}
\usepackage{amssymb}
\usepackage{mathtools}
\usepackage{graphicx}
\usepackage[T1]{fontenc}
\usepackage[utf8]{inputenc}
\graphicspath{{images/}}
\usepackage{color}
\usepackage[pdfstartview=FitH,
            breaklinks=true,
            bookmarksopen=false,
            bookmarksnumbered=true,
            colorlinks=true,
            linkcolor=black,
            citecolor=black,
            urlcolor=black,
            pdftitle={Peptidome},
            pdfauthor={Andreas Mayer},
            pdfsubject={}
            ]{hyperref}
\newcommand{\B}{\boldsymbol}
\newcommand{\ud}{\mathrm{d}}
\newcommand{\<}{\langle}
\renewcommand{\>}{\rangle}

\def\(({\left(}
\def\)){\right)}                       
\def\[[{\left[}
\def\]]{\right]}

\newcommand{\AM}[1]{{\color{blue}#1}}

\begin{document}

\title{A statistical ensemble approach to immune discrimination}
\author{Andreas Mayer}
\author{Quentin Marcou}
\author{William Bialek}
\date{\today}

\begin{abstract}
    Can we view the self/non-self discrimination problem in a statistical language?
    We explore whether the self-peptidome and the peptidome of pathogens differ in a statistical manner.
\end{abstract}

\maketitle

\section{Ideas}




\bibliography{library}

\end{document}


