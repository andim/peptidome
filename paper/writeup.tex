\documentclass[superscriptaddress,onecolumn,pre]{revtex4}
\bibliographystyle{apsrev}

\usepackage{ifthen}
\newboolean{pnas}
\setboolean{pnas}{false}

\usepackage{amsmath}
\usepackage{amsfonts}
\usepackage{amssymb}
\usepackage{mathtools}
\usepackage{graphicx}
\usepackage[T1]{fontenc}
\usepackage[utf8]{inputenc}
\graphicspath{{images/}}
\usepackage{color}
\usepackage[pdfstartview=FitH,
            breaklinks=true,
            bookmarksopen=false,
            bookmarksnumbered=true,
            colorlinks=true,
            linkcolor=black,
            citecolor=black,
            urlcolor=black,
            pdftitle={Peptidome},
            pdfauthor={Andreas Mayer},
            pdfsubject={}
            ]{hyperref}
\newcommand{\B}{\boldsymbol}
\newcommand{\ud}{\mathrm{d}}
\newcommand{\<}{\langle}
\renewcommand{\>}{\rangle}

\def\(({\left(}
\def\)){\right)}                       
\def\[[{\left[}
\def\]]{\right]}

\newcommand{\AM}[1]{{\color{blue}#1}}

\begin{document}

\title{A statistical ensemble approach to immune discrimination}
\author{Andreas Mayer}
\author{Quentin Marcou}
\author{William Bialek}
\date{\today}

\begin{abstract}
    Can we view the self/non-self discrimination problem in a statistical language?
    We explore whether the self-peptidome and the peptidome of pathogens differ in a statistical manner.
\end{abstract}

\maketitle

\section{Ideas}

Are there any features that distinguish foreign antigens from self-antigens? There is one view of adaptive immunity in which both self and non-self antigens are random samples from a common (and essentially flat) distribution of peptides of a given length. Discrimination is then achieved solely on the basis of "white-listing": thymic negative selection acts to get rid of those cells that are reactive to self, leaving everything else as potentially foreign. If the two types of antigens are instead drawn from different distributions, than some regions of antigenic space will be much more likely to be self and some much more likely to be non-self. Over evolutionary timescales the recombination machinery might then have evolved to bias the immune repertoire towards recognizing antigens that are more unlikely to arise from the human proteome.

Maybe the immunogenicity of an antigen is related to how untypical it is given the normal distribution of the human proteome. This could be a useful insight for cancer immunotherapy: Are good neoantigens those that represent large perturbations from the normal distribution? This could also be of relevance for autoimmunity, where more uncommon peptides within the self-peptidome might also be more likely to lead to autoimmunity.

It could be interesting to also consider abundance information (see e.g. \url{
    https://pax-db.org/}), i.e. weigh a peptide by the abundance of the protein from which it arises. Again in immunology \cite{Walz2015}



\bibliography{library}

\end{document}


